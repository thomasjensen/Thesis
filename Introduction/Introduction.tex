\chapter{Introduction}
\lettrine[lines = 3]{T}{he} Council of minister (hereafter referred to as the 'Council') is at the center of the European Union (EU). All legislation must be adopted in the Council before it can enter into force. In much of the early literature on the Council it was considered to be the playground of member states with governments fighting for their 

The council as the most importan institution.

The EP as a co legislator

The importance of EU legislation, hence the importance of understanding the Council

How does the Council work

What we know so far

Critique

What is the purpose of this book? research question!

research design




The literature on the Council of ministers has within the last couple of years reached a level of maturity where we know have three (almost) universally accepted stylized facts.

\begin{enumerate}
\item Council decision making is structured by a wish to reach consensus.
\item There are weak cleavages that structure conflict (north-south, new-old, left-right)
\item The presidency matters
\end{enumerate}

Of these stylized facts the least tested is the claim that a norm of consensus has a strong causal effect on decision making in the Council. The common argument for the consensus norm has three key components. First actors interact with each other in an insulated environment. Secondly the population of negotiators change very slowly, so actors can build up long histories of interactions. Thirdly, this allows for the development of diffuse reciprocity. Once diffuse reciprocity is present, then the development of a consensus norm follows closely. 

The evidence for a consensus norm comes from three sources, namely case studies, council voting records and the DEU data set. The use of different kinds of data, not surprisingly, impacts the conclusions reached [WRITE MORE]. However the conceptualization of the consensus norm also differ across approaches, with the quantitative studies often employing a simple indicator for consensual voting as evidence for the consensus norm. In contrast many case study scholars work with richer conceptualizations of the consensus norm. For them a consensus norm not only implies consensual voting, but consensual voting is seen as an outcome of the effect the consensus norm has on the interaction among negotiatiors, and for many scholars this is the interesting effect. Although there is a clear link between the consensus norm and consensual voting it is not perfect. Consensual voting can be caused by either the consensus norm or log-rolling, when only studying voting outcomes it is difficult to disentangle these two causes from the effect, whereas this is possible when conducting case studies. 

The scholars who study decision making through case study methods present the most consistent evidence for a consensus norm. Table \ref{tab:casestudies} presents an overview of the cases, the studies in which they have been used, and the main conclusions from the studies. 

\begin{table}[htp]
  \centering
  \begin{tabular}{p{3cm} l p{3cm}  p{3cm} p{3cm}} \hline
    Case & Year & Policy Area & Studies & Findings \\ \hline
    Local Elections Directive & 1994 & Legal Affairs & Lewis 1998, Lewis 2003, Lewis 2005 & Strong effect of the consensus norm \\
    Working Time Directive & 1993 & Employmant and Social Affairs & Lewis 2003 & Consensus norm was trumped by domestic concerns \\
    Internal Energy Market Directiv & 1996 & Internal Market & Eising 2002 & The consensus norm work in conjunction with fomal institutions \\
    Transparancy Regulation & 2001 & Legal Affairs & Elgstr\"{o}m \& Bjurulf & The consensus norm work in conjunction with formal institutions \\
   Dublin II Regulation & 2003 & Justice and Home Affairs &Aus 2008 & Strong effect of the consensus norm \\ \hline 
  \end{tabular}
  \caption{Case Studies and the Consensus Norm}
  \label{tab:casestudies}
\end{table}


All authors agree that the consensus norm is present in the Council; they also agree that the effect of the norm is to produce a cooperative negotiation style; however they they differ with regards to how much causal efficacy is given to the norm. Lewis is the strongest proponent for attributing a strong causal effect to the consensus norm. In his case studies the norm is the main causal explanation for the the reason why Austria and Belgium did not block the local elections directive. The decision not to push for a vote, and the willingness of the Council to accomodate the concerns of Belgium is interpreted by Lewis as evidence for a culture of compromise. The working time directive from 1993 illustrates that the consensus norm is not universally in effect, but does vary according to certain scope conditions. In the light of strong domestic pressure and public scrutiny the consensus norm breaks down. The study by Eising illustrate that one effect of the consensus norm is to restrain the set of feasible actions available to negotiatiors. However only in conjunction with an efficient institutional capacity for resolving conflict and facilitate policy learning does the consensus norm lead to political agreement in the light of conflict. This is also confirmed by Elgtrom and Bjurulf. In their study the consensus norm only works in some conditions. The French presidency, that precided over the first stage of the negotations, took a decidedly majoritarian approach, catering to the qualified majority and making minimal consessions to the minotrity. However due to the involvement of the European parliament, a vote was not pushed. In the second half of the negotiation Sweden, belonging to the minority block, took over the presidency and brokered an agreement. The role of the consensus norm was visible in the willingnes of the minority to accept the presidency compromise, and the unilateral concessions offered by several memberstates. However the effect was hampered due to the involvement of the EP, which was threatening with a possible conciliation round. The detailed study by Aus on the Dublin II regulation show how the consensus norm can be used strategically to overcome opposition. Greed and Italy where staunchly opposed to the principle of first contact\footnote{Asylum seekers are to have their application processed in the member state though which they first enteres the EU.}. Throughout the Belgian and Spanish presidencies the greek and italian reservations stood firm. Proposals for altering the first contact principle stranded on opposition from Germany and France, thus there where no end to the stalemate in sight. During the Danish presidency presented the Council with a political decleration promising solidarity with member states disproportional burdened by asylum applications, this statement was attached to the Dublin II regulation, but failed to convince Greece and Italy. In a last move the Danish presidency deicded to adopt a silent procedure, putting the proposal along with the political decleration forward to the Council: While a written procedure requires member states to explicitly state their agreement, a silent procedure requires member states to explicitly state their opposition. Thus since no member state wanted to ``rock the boat'' , no opposition was recorded, and a political agreement was reached. 

The case studies detailed above all present compelling evidence for the presence of a consensus norm, however they also present a very biased case selection. As table \ref{tab:casestudies} show there are very few cases covering a very limitecd range of policy areas. Furthermore the cases cover a time span from 1993 to 2003. This provokes the question of what we can learn from these cases. The internal validity of all the cases are not in question, but to what degree these cases represent typical negotiations, or are in some respect outliers is not clear. Furthermore only in two cases do we find any independent causal effect of the consensus norm, in the rest of the cases the consensus norm only carry causal efficacy in conjunction with other causes. 


The DEU data has in many regards allowed scholars to test classic hypotheses from the rational-institutionalisst litarature in a new and more comprehensive manner. One of the conclusions of the project was that 








If one wants to study the consensus norm over a large time period, and have a large number of cases, the voting records from the Council minutes must be used. 





However using these records to study decision making in the Council is not unproblematic.  Most studies using the voting records from the Council focus either on identifying the conflict structure of the Council, i.e. which cleavages seem to determine the voting behavior, or on explaining voting behavior in the aggregate (such as the number of bi-annual no votes and abstentions). There has been several critiques of this approach. Hagemann points out that votes are not the only information on conflict in the Council. Member states often make statements as reactions to votes in the Council, and ignoring the information present in these statements risks biasing the results towards finding consus. Furthermore most analyses infer their conclusions with regards to the member state, however as Hagemann and Hoyland point out, the Council is more correctly viewed as an assembly of different cabinets. It is concievable that different cabinets from the same member state will behave very differently in the Council, thus we risk making faulty inferences about voting behavior if we gloss over the different cabinets of a member state. 

With regards to the unit of analysis, many studies use aggregate data to make inferences about the Council as an entity, whereas the argument about the consensus norm works on the individual negotiator in the Council, thus there is often a mismatch between the theoretical unit of analysis, and the empirical unit of analysis. Only Hagemann and Hoyland (2008, 2010) have broken down the council vote to their natural unit of analysis, namely the single vote of a given cabinet on one dossier, however they do not model the decision to vote yes, no or abstain directly as a function of a set of independent variables.

There are also problems with the case selection in many studies of the consensus norm. In most of the case studies it is not clear how the case selected stands in relation to the population of relevant cases, indeed often the relevant population is not properly defined. This makes it difficult to generalize beyond the single case. Studies using the voting records from the Council suffer from a different type of selection bias. often studies either include only the distinction between definitive and ther legal act to seperate relevant from irrelevant cases. Depending on the research question this can be a valid approach, however when studying the consensus norm this procedure risks biasing the case selection. It is important to distinguish controversial dossiers from non-controversial dossiers. When there is no controversy, there is no need to seek consensus actively, thus including non-controversial acts in an analysis risks biasing it towards finding consensus. The distinction between definitive an other acts is not appropriate as it is possible to have non-controversial cases amon the definitive legal acts, and one can find controversial acts among the other legal acts. Some studies use the voting records as an indicator for when an act is controversial or not, but this is not appropriate when the dependnet variable is the voting behavior in the Council. 