
\chapter{Determinants of Vote Choice}
\label{chapter:votechoice}

\lettrine{T}{he} Food Flavoring directive adopted in 2008 was a controversial dossier. Especially Germany had issues surrounding the levels of allowed additives derived from aromatic spices and herbs. The dossier was adopted unanimously in the Council, but the German government attached the following statement to their vote. 

\begin{quote}
  ``Germany would have preferred a lower coumarin limit for traditional and seasonal bakery ware containing cinnamon the the one now provided for, but in a spirit of compromise can nevertheless support the proposed maximum level''
\flushright{ - Council Monthly Summaries October 2008 p. 21}
\end{quote}

 The dossier had been debated in the Council and parliament since 2006 and the parliament had significantly strengthened the scope of text from the Commission. Among other things, the Parliament proposed to leave an annex regarding the risk of using certain aromatic herbs and spices open until further evidence could be provided to determine the appropriate risk levels. This was welcomed by Germany which had argued that the levels set in the Commission proposal with regards to cinnamon were too high. However the Parliaments amendments was met with opposition from other governments in the Council, who feared that traditional artisans that produced bakery goods with cinnamon would be unnecessarily burdened by a lower limit.  Recognizing that the Council could not reach a decision the Commission withdrew the proposal, to redraft it. After several debates in the Council Germany agreed to give in on its reservations. 

The voting behavior of Germany on the Food Flavoring directive represents a case where  ....

This sort of behavior is an aspect of decision making not captured in current studies of the Council.  Most studies so far only examine yes votes, no votes or abstentions, thus situations like the one above where Germany expressed strong reservations about the legislation, but still voted yes, are not captured. Including cases like the one above thus allows us to study voting behavior in a more fine grained manner.

It is a commonly used explanation whenever a pattern of consensus is found in the Council to refer to the consensus norm, and often the high degree of unanimously adopted dossiers in the Council is used as corroborating evidence. However using unanimously adopted decisions as evidence for a consensus norm is not without pitfalls. There are many reasons for why a dossier was adopted unanimously. A dossier could be uncontroversial, thus requiring no need for seeking consensus. It is also possible that the preferences where so aligned that an agreement was reached without needing to resort to long negotiations. Finally it might be that logrolling was the mechanism. 

The cases of inflating a data set with uncontroversial dossiers or dossiers where the preferences were aligned can be remedied at the case selection stage. To distinguish between logrolling and diffuse reciprocity it is argued that behavior of the type exhibited by Germany above can be interpreted as engaging in diffuse reciprocity, by building up the ``favor bank''. 


\section{Diffuse Reciprocity and Voting in the Council}

Social norms offer consideral scope for interpretation, manipulation and choice. To treat norms as monolithic entities that determine social interactions is to revert back to the fallacy of the over-socialized actor that has plagued much of constructivist theory, however to treat norms purely n as tools to mask self-interest is to make the fallacy of the under-socialized actor plaguing much of rational choice theory \citetext{see \citealt{Elster1989} and \citealt{Schimmelfennig2000} for good overviews and critiques}. The theory of diffuse reciprocity is based around the notion of unanticipated consequences, as exchanges aquire a decidedly social flavor, actors get caught up in the emerging system of debts and credits. It is important to keep in mind that at no point are actors aiming towards such a system, it is an emergent phenomena resultant from the interactions in a social system. Since diffuse reciprocity do not rely on the internalization of a social norm, it is closer to a social equilibrium in which most actors would be worse of if they engaged in off equilibrium behavior. In such a system it is always a possibility that a government choose to pursue its own interest in hope of some short term gain, if the expected gain is higher than the expected long term loss in standing. The pertinent question raised here is when will a government decide to deviate from the exchange system, and pursue its own narrow interests. Drawing upon the work in chapter \ref{chapter:theory} we can label the incentives to defect in two categories, namely domestic concerns and preferences. These two categories are not entirely mutually exclusive, it is possible that domestic concerns will align with strong preferences against the final outcome. Furthermore the two categories are roughly correspondent to the classic concerns over whether politicians are mainly driven by the wish to succeed electorally, or to influence policies \citep{Strom1990}. There is evidence for both types of motivations in the literature on coalition formation, for instance \citet{MartinStevenson2001} show that parties, when deciding with whom to create a coalition, will show concerns for which office they will hold and the policy content of the coalition. In the Council concerns about electoral success can manifest themselves when domestic politics and EU legislation collide. This can often put governments in delicate situations where they must choose between being good members of the EU and appeasing domestic constituencies. The trouble of the reintroduction of the border control at the Danish border is illustrative in this regard. As part of a domestic settlement the Danish government agreed to introduce a limited border control in order to get the votes from the nationalistic Danish Peoples Party (DPP). Since Denmark is a member of the Schengen area, the implementation could only be very limited if should be complient with Schengen rules. However at home the DPP was claiming that the border controls would be a return to the border control that was in place befor Denmark joined the Schengen area \citep{Olsen2011,Euractiv2011}. This put the government in a difficult position where it had to choose between either downplaying the role of the border control, which would offend the DPP, or going along with the DPPs claim which would offend the Commission. 

Governments in the Council face two types of domestic constraints. One type can be labelled interest group pressure, and is distunguished by strong interest groups pushing an agenda at the EU level that is not necessarily in line with the preferences of the government. Lobbying has long been part of negotiations in Brussels. Lobbying typically display a pattern where national interests, organized in an umbrella organization, will go through their European counterpart to influence policy makers in Brussels. However often national umbrella organizations will also contact the relevant national ministers directly in order to convince them to push an agenda in the Council \citep{Beyers2002}. Most often interest groups will focus their attention on changing parts of the legislative text that they are not happy about, however when these efforts have proven futile the focus might be on recommending a government to vote no as a signal of support for domestic concerns. This has the effect of building relations between key constituent groups domestically, but will lower the standing of the government in the Council. The other constraint can be labelled public opinion. If a dossier at the EU level catches the eye of the public in a member state, it can have an impact in the re-election concerns of governments, particularly on EU sceptic member states such as the UK and Denmark. An interesting dynamic that has often been observed in the Council is blame shifting. In multilevel systems it is easy for institutions at at given level to engage in blame shifting and credit taking for political outcomes \citep{Anderson2006}. This has been observed in member states where governance is divided between different administrative structures, such as France and Germany \citep{Elgie2006}. It is no surprise that the same dynamic has also been oberserved to take place between member states governments and the EU \citep{Franchino2004}. If governments find themselves under pressure domestically, but are unable to block legislation at the EU level, it is tempting to vote no in order to appease the domestic audience, and blame the EU for forcing unwanted legislation. 

In the absence of domestic constraints, governments migh find themselves highly disagreeing with the outcome. The theory of diffuse reciprocity will predict that in these situation, the vote choice is dependent upon how much the future is weighted. If an actor has been very active in maintaining exchange relations then the future is weighted heavily, thus we would expect this actor to compromise. However actors that are not active in the system of exchanges will not weight the future very heavily, hence we will expect these actors to vote more along their preferences. This line of argument has been developed by \citet{HoylandHansen2010}. They argue that the consensus norm in the Council can be viewed as the propensity to override once own preferences in favor of a consensual decision. They are thus in line with the direct indicator approach specified in chapter \ref{chapter:theory}. In order to further specify this mechanism the authors use a spatial model with an added term to capture any effect of a consensus norm. Assuming that actors operate in a one-dimensional space with a linear loss function, the principle of utility maximization implies that an actor $i$ will vote yes to a dossier if:

\begin{equation}
  \label{eq:1}
  U_{ik}(-|x_{ik} - p_k|) > U_{ik}(-|x_{ik} - q_k|)
\end{equation}

Where $U_{ik}$ is the utility of actor $i$ on dossier $k$, $x_{ik}$ is the ideal position of actor $i$ on dossier $k$, $p_k$ is the position of the proposed dossier and $q_k$ is the status quo. Equation \ref{eq:1} thus represents the classic spatial model, to account for the effect of the consensus Hoyland \& Hansen add a third term, $\theta$, which is added in linear fashion to the loss function. Hence:

\begin{align}
  \label{eq:2}
  U_{ik}(-|x_{ik} - p_k|) < U_{ik}(-|x_{ik} - q_k|) &= No \\
  U_{ik}(-|x_{ik} - p_k| + \theta_i) > U_{ik}(-|x_{ik} - q_k|) &= Yes
\end{align}

Theta, if positive, will have the effect of reversing the vote, whereas if set to zero or a negative value will have no effect. According to \citet{HoylandHansen2010} if theta is constant across actors this will be evidence of a norm, however the theory of diffuse reciprocity will expect that there is variation in theta, as it more accurately reflects the degree to which a given actor is embedded in an exchange system. The modified spatial model presented by Hoyland \& Hansen helps to illuminate some of the dynamics 

a social dilemma: vote along with preferences for an immediate benefit, or vote against for a diffuse long term benefit. 


The role of preferences and salience in diffuse reciprocity

The role of the social environment

what determines vote choice


\section{Research Design}

\subsection{Case Selection}

In order to test competing hypotheses about voting behavior in the Council it is important to have the proper universe of cases. As detailed above, hypotheses about the consensus norm are only interesting in cases where the norm has the possibility to assert causal efficacy. Likewise rational-institutionalist hypotheses are only interesting in cases where it is possible for preferences and institutions to affect the outcome. Hence dossiers in which there was no controversy, i.e. no political involvement was necessary, are not interesting. Indeed including these cases in the analysis risks biasing the analysis towards finding no effect of the variables. Therefor the first step in the analysis is to define the population of relevant cases. In our case the population is defined as dossiers treated in the Council where there was disagreement about the final outcome. In the literature in decision making in the EU several criteria has been used to separate controversial from uncontroversial dossiers. It has been common in the voting record literature to analyze cases on which there was a dissenting vote separately \citep{Heisenberg2005,Hayes-renshaw2006,Hagemann2008}. However this introduces selection bias if the dependent variable is the vote choice of the member states. The DEU project \citep{Thomson2006a} used the criteria of public reporting. If an act was mentioned in Agence Europe it was considered as being controversial. This approach has the drawback that it is not possible to discern whether this controversy represented divisions within the Council, or disagreements between constituent groups and the decision reached by the Council. In order to avoid this ambiguity the present study uses a more direct measure of controversy within the Council. The cases used in this paper are selected on the basis of having, at one point, been treated as a B point in the Council. The Council agenda distinguishes between A and B points, where A points are dossiers that do not require discussion at the ministerial level, and B points are dossiers in which no agreement could be reached on working group level and need discussion at ministerial level. The advantage of selecting only cases treated as B points in the Council is that we are sure to select cases in which there were some controversy related to differences within the Council, and the selection rule does not restrict the variance on the voting outcome. 

\subsection{Data and Variables}
The dataset used in this paper cover the years 1999 to 2009 and includes every vote in the Council during this time period. In total there are votes registered on 3937 dossiers, which gives a total of 78968 votes cast by member states governments. Of these 16473 where cast for dossiers that at one point was a B point in the Council. Beside the vote cast, the data also contains information on whether a statement was registered by any member state, and whether this statement expressed dissatisfaction with the outcome. This coding is more fine grained compared to most studies who either ignore the possibility of registering a statement in conjunction with a vote, or simply code whether there was a formal statement or not.

The focus of this study is the voting behavior of the member states in the Council of the EU. As such the dependent variable is the individual vote choice of the government in power in a given member state. There are three vote choices available to any member state, namely yes, no or abstain. These choices are coupled with the possibility of also registering a statement in conjunction with the vote. Thus there are six possible combinations of vote and statement:

\begin{enumerate}
\item Yes
\item No
\item Abstain
\item Yes + Statement
\item No + Statement
\item Abstain + Statement
\end{enumerate}

Here we have only included the statements that express dissatisfaction with the final outcome. Thus the dependent variable reflect one category of agreement (yes) and five categories of different types of dissent. It is difficult to rank vote choices, they represent different choices which do not have a well defined high and low point according to which they can be ranked. For this reason the variable is treated as being nominal in this paper. 

The independent variables are measured on two levels, namely the dossier level and the member state government level. The member state government is the lowest level in the data, and are nested with the dossier level, such that for each dossier we have recorded between 15 and 27 votes that are attributed to governments from the member states. A potential biasing factor is the presence of country specific interests in given policy areas. Without very detailed knowledge of the economies of each member state it is difficult to device a measure of national interests across policy areas. Given that the simplest way to control for national interests is to include dummies coding for whether a member state has a large interest in a policy area,  an equivalent approach is to include both country and policy area dummies in the model. However due to  some member state governments not utilizing all voting choices during their stay in office, including country dummies leads to many empty cells, and hence the model cannot converge. Hence the best we can do in order to see if there are any biasing effects is to first conduct an analysis on the full data, without country dummies. Then conduct a separate analysis on a subset of the data only including member states that utilized all voting choices. This is not an ideal solution, but since national interest are only needed as a control it is acceptable in this context. This analysis was conducted and the effects did not change substantially, thus in the analysis below only the Council configuration dummies along with the dummy for dossiers dealing with agriculture and regional policies are included in the models. 

\subparagraph{Left-right} The position of each government was recorded on a 0 tp 10 scale where 0 is the most left leaning a government can be, and 10 is most right leaning a government can be. The data was taken from the ParlGov database (Doring Mannow 2010). Since the variable is recorded on the government level, the same information is also used to calculate the median position with in the Council at any given vote. From this the distance between any government and the Council median was calculated

\subparagraph{Power} To distinguish powerful governments from the less powerful, the raw vote count of a given member state is used. The vote count is based on the population size of a member state, and correlates highly with GDP. The vote is not transformed into a power index as these tend to produce the same ranking as the raw vote count for the Council [REFERENCES] and in order to make the variable more interpretable the raw vote count is used. 

\subparagraph{Presidency} A simple categorical variable coding for which government is holding the presidency is included.

\subparagraph{Net Beneficiary} This variable takes the net amount a member state received from the EU in a given year and subtracts it from the amount paid to the EU. A positive value indicates a net beneficiary, whereas a negative value indicates a net contributor. 

\subparagraph{Agriculture/Regional Policies} A dummy variable distinguishing whether a dossier fell under agriculture or regional policy areas.

\subparagraph{Meeting Frequency} The potential for socialization within a given Council configuration is measured through the number of Council sessions held in a given year. One alternative would be to count the number of Council sessions between the Commission submission of the dossier to the Council and the final vote. However since the socialization argument is cast at the Council level, and not the single dossier, it is more appropriate to use the yearly count of meetings as a proxy for interaction density. 

\subparagraph{Voting Rule} A categorical variable controlling for whether unanimity or QMV was in force in the final vote on a given dossier.

\subparagraph{Type of Legal Act} A categorical variable controlling for whether the Council was voting on a directive, regulation, or decision. 

\subparagraph{Salience} A count variable controlling for the number of times a dossier was discussed at ministerial level. This is not a country specific salience measure, but a measure of how salient the dossier was for the Council as a whole. 

\subsection{Method}

As the dependent variable takes the shape of a nominal variable a multinomial logit is chosen. The multinomial logit model is specifically designed for modeling variables that take the shape of different, non-ranked, choices, also referred to as revealed preferences. The advantage og using a multinomial model over an ordinal is that we do need to assume any type of ranking in the dependent variable, and hence when there is doubt about how to rank a variable this approach is more suitable. Multinomial models are complex beasts, and require care when estimating and interpreting. Since the  model is essentially a series of logit models estimated simultaneously, one for every pair of choices, we have many equations to keep track of \citep[CHapter 6]{Long1997}. It is usual to choose one baseline category, and then estimate all models with this baseline as the reference. Here I have chosen to use the yes category as the baseline, and estimated the 5 equations pairing every other category with the baseline. Thus when interpreting the coefficients in the models, they represent the effects of choosing, say to abstain, compared to choosing to vote yes. 



\section{Descriptive Evidence}
As a first look at the data table \ref{tab:depvar} show the distribution of votes across Council Configurations. The number of yes votes dominates the distribution. Even after having selected only definitive legal acts that where treated as B points in the Council, the number of yes votes is staggering. Surprisingly the second largest category is yes votes that have a dissenting statement attached. This category is in all Council configurations the second largest, and show that studies not taking Council statements into account risk biasing their findings towards consent. The prevalence of attaching negative statements to supporting votes suggests that there is a large degree of signaling behavior going on in the Council. It is important to let your colleagues know that you are a team player and willing to compromise. This speaks fot the interpretation of negotiations in the Council as oriented towards diffuse reciprocity and problem-solving. It is also striking that dissenting behavior is not equally distributed across Council configurations. Agriculture\& Fisheries, Competition and Employment, Health \& Consumer Affairs exhibit a very high degree of dissent when compared to the other Council configurations. Thus we clearly see some policy specific effects on voting.


\begin{table*}[ht]
\begin{center}
\resizebox{\textwidth}{!}{
\begin{tabular}{rrrrrrrr}
  \hline
 & 1 & 2 & 3 & 4 & 5 & 6 & Total\\ 
  \hline
Agri/Pech & 2760 &  92 &  41 &  35 &  17 & 160 & 3105\\ 
  Comp. & 2993 &  20 &  16 &  16 &   7 &  42 & 3094\\ 
  Ecofin & 247 &   0 &   0 &   0 &   0 &   0 & 247\\ 
  Education/Culture & 246 &   2 &   0 &   0 &   2 &   9 & 259\\ 
  Employment/Consumer & 1896 &  26 &  20 &  23 &  18 &  55 & 2038\\ 
  Environment & 1216 &  16 &   6 &  15 &   8 &  20 & 1281\\ 
  Gen. Affairs & 227 &   5 &   0 &   0 &   0 &   9 & 241\\ 
  JHA & 1192 &   2 &   1 &   1 &   2 &  21 & 1219\\ 
  Trans./Energy & 1975 &  12 &  11 &   8 &   3 &  34 & 2043\\ \hline
Total & 12752 & 175 & 95 & 98 & 57 & 350 & 13527\\
   \hline
\end{tabular}
}
\end{center}
\caption{The Distribution of Votes Across Council Configurations}
\label{tab:depvar}
\end{table*}

Table \ref{tab:dissent} lists the ten most dissenting governments in the period 1999 - 2009. The governments were ranked according to how many dissenting votes they had in the Council divided by the number of days they spend in government. The left-right position of the government, the average Council left-right score during the governments stay in office and the deviance between the two scores is also reported. A common finding in the literature on the Council is that large member states are more prone to vote no or abstain than small member states. However if we disaggregate the data into member state governments and adjust for time spend in power, then a different picture emerges. Austria is one of the member states that is usually regarded as being not prone to dissent, but during the short lived second Schuessel cabinet, the government had a series of conflicts in the Council. Moreover the table gives a first hint that there might be an effect of the left-right position of a government on voting behavior in the Council. Except for Schuessel, Verhofstadt and Balkanende, all governments deviate more than one point from the mean Council position, in the case of Schroeder, Persson and Marcinkiewicz more than two points. There are also some hints that member state size, left-right deviation and dissent are related. 

\begin{table*}[ht]
\begin{center}
\resizebox{\textwidth}{!}{
\begin{tabular}{ll c c c p{3cm} p{3cm} p{3cm}}
  \hline
 Country & Cabinet & Ratio & Dissent & Days in Power & Left-Right & Mean Council Left-Right & Deviance \\ 
  \hline
ITA &D'Alema\_I & 0.06 &  25 & 426 & 2.63 & 4.53 & 1.90 \\ 
 AUS & Schuessel\_II & 0.04 &   4 & 96 & 5.20 & 5.75 & 0.55 \\ 
DEN & Rasmussen\_F I & 0.02 &  28 & 1179 & 7.22 & 5.71 & 1.51 \\ 
POL & Marcinkiewicz\_I & 0.02 &   4 & 186 & 7.75 & 5.39 & 2.37 \\ 
BEL & Verhofstadt\_III & 0.02 &   4 & 194 & 5.69 & 5.42 & 0.27 \\ 
SWE & Persson\_III & 0.02 &  28 & 1446 & 3.37 & 5.64 & 2.28 \\ 
GER & Schroeder\_II & 0.02 &  21 & 1127 & 3.24 & 5.69 & 2.46 \\ 
UK & Blair\_II & 0.02 &  26 & 1428 & 4.18 & 5.65 & 1.47 \\ 
NET & Balkenende\_III & 0.02 &   4 & 230 & 5.26 & 5.47 & 0.21 \\ 
FRA & Raffarin\_III & 0.02 &   7 & 427 & 6.80 & 5.61 & 1.20 \\ 
   \hline
\end{tabular}
}
\end{center}
\caption{The Ten Most Dissenting Governments Between 1999 and 2009}
\label{tab:dissent}
\end{table*}

In order to see if the presidency effect is present in the data, the data was split into two groups, one group consisted of all votes in which the government that voted held the presidency, and the other group consisted of all votes in which the government did not hold the presidency. In each group the number of dissenting votes relative to the total number of votes was calculated, and surprisingly there were only minuscule differences between the ratios (0.057 for the presidency group and 0.058 for the non-presidency group). Thus on the aggregate level the presidency does not seem to reduce the level of dissent. However if we examine individual governments there are some differences. When only looking at no votes and abstentions (with and without statements), the Blair II government see a reduction from 0.03 to 0 when holding the presidency. However if we include yes votes with a dissenting statement as part of the dissent category the presidency increases the level of dissent from 0.09 to 0.14. Indeed only examining the yes votes with a dissenting statement, there is an increase from 0.06 to 0.14. Thus there are large effects for some governments, however for most governments this is not the case.


\section{Inferential Analysis}

In order to further test whether the relationships found here hold when controlling for potentially confounding effects, and to test the other hypotheses, a multinomial choice model is used. Table \ref{tab:summary} presents summary statistics for the continuous and count variables used in the model, the categorical variables are excluded. The Hausman test for irrelevant alternatives rejected the null hypothesis, thus the IIA assumption is not violated. Tables a - d in panel 4 show the results from the models. All models include Council configuration dummies, to save space these are not reported. Table a only contain the variables of theoretical interest, tables b - d introduce controls for dossier type, voting rule and salience, table 5 show the full model with all controls Finally table 6 show the full model with all controls, including an interaction term between the power and outlier variables. When including an interaction term in a statistical model the constituent parts of the term can no longer be meaningfully interpreted in their own right. Therefore, in order to be able to evaluate hypotheses one and two, the interpretation will focus on the full model with and without the interaction term.

\begin{table*}[hrp]
\subfloat[Model with Council Dummies]{
\resizebox{\textwidth}{!}{
\begin{tabular}{rrrrrrrr}
  \hline
 & Constant & Outlier & Power & Net Beneficiary & Meeting Frequency & Presidency & Agriculture/Regional \\ 
  \hline
1$|$2 & 2.29 & -0.22 & 0.01 & 0.00 & 0.06 & 0.07 & -0.06 \\ 
          & 5.57 & -2.94 & 1.52 & 1.99 & 1.83 & 0.27 & -0.36 \\
1$|$3 & -0.99 & 0.02 & 0.03 & -0.00 & -0.02 & 0.24 & 0.82 \\ 
        & -1.37 & 0.19 & 1.83 & -0.14 & -0.27 & 0.59 & 3.31 \\
  1$|$4 & -0.78 & -0.06 & 0.03 & 0.00 & -0.07 & 0.18 & -0.12 \\ 
            & -0.88 & -0.34 & 2.08 & 0.48 & -0.90 & 0.34 & -0.34 \\ 
  1$|$5 & -1.07 & 0.03 & -0.03 & 0.00 & -0.04 & -0.20 & 0.47 \\ 
             & -1.24 & 0.18 & -1.42 & 1.28 & -0.52 & -0.36 & 1.50 \\ 
  1$|$6 & -0.12 & -0.05 & -0.01 & 0.00 & -0.17 & -0.31 & -0.11 \\ 
             & -0.10 & -0.27 & -0.55 & 0.11 & -1.68 & -0.41 & -0.24 \\ 
   \hline
\end{tabular}
}
}

\vspace{1em}

\subfloat[Model with Council Dummies and Controlling for Dossier Type]{
\resizebox{\textwidth}{!}{
\begin{tabular}{rrrrrrrrrr}
  \hline
 & Constant & Outlier & Power & Net Beneficiary & Meeting Frequency & Presidency & Agriculture/Regional & Directive & Regulation \\ 
  \hline
1$|$2 & 3.01 & -0.22 & 0.01 & 0.00 & 0.05 & 0.07 & 0.00 & -0.43 & -0.57 \\ 
& 6.18 & -2.97 & 1.56 & 1.98 & 1.31 & 0.26 & 0.00 & -2.08 & -2.93 \\ 
  1$|$3 & -0.10 & 0.02 & 0.03 & -0.00 & -0.03 & 0.24 & 0.89 & -0.82 & -0.78 \\ 
& -0.13 & 0.18 & 1.86 & -0.16 & -0.49 & 0.59 & 3.53 & -2.39 & -2.44 \\ 
  1$|$4 & -0.39 & -0.07 & 0.04 & 0.00 & -0.11 & 0.17 & -0.03 & 1.23 & 0.05 \\ 
 & -0.36 & -0.41 & 2.11 & 0.51 & -1.45 & 0.33 & -0.08 & 2.36 & 0.09 \\ 
  1$|$5 & -0.41 & 0.03 & -0.03 & 0.00 & -0.05 & -0.20 & 0.52 & -0.56 & -0.56 \\ 
& -0.41 & 0.17 & -1.41 & 1.27 & -0.66 & -0.36 & 1.62 & -1.45 & -1.45 \\ 
  1$|$6 & -0.91 & -0.05 & -0.01 & 0.00 & -0.14 & -0.31 & -0.18 & 0.21 & 0.49 \\ 
 & -0.65 & -0.25 & -0.55 & 0.11 & -1.33 & -0.40 & -0.41 & 0.38 & 0.90 \\ 
   \hline
\end{tabular}
}
}

\hspace{1em}

\subfloat[Model with Council Dummies and Controlling for Voting Rule]{
\resizebox{\textwidth}{!}{ 
\begin{tabular}{rrrrrrrrrr}
  \hline
  & Constant & Outlier & Power & Net Beneficiary & Meeting Frequency & Presidency & Agriculture/Regional & QMV & Unanimity \\ 
 \hline
 1$|$2 & 2.74 & -0.22 & 0.01 & 0.00 & 0.06 & 0.07 & -0.07 & -0.46 & -0.63 \\ 
& 4.83 & -2.93 & 1.50 & 2.00 & 1.88 & 0.27 & -0.45 & -1.18 & -1.53 \\ 
  1$|$3 & -1.64 & 0.03 & 0.03 & -0.00 & -0.01 & 0.24 & 0.74 & 0.57 & -1.00 \\ 
& -1.50 & 0.23 & 1.78 & -0.14 & -0.09 & 0.60 & 2.99 & 0.70 & -1.06 \\ 
  1$|$4 & -1.62 & -0.05 & 0.03 & 0.00 & -0.06 & 0.19 & -0.20 & 0.83 & -1.34 \\ 
& -1.15 & -0.31 & 2.03 & 0.48 & -0.84 & 0.36 & -0.58 & 0.76 & -1.03 \\ 
  1$|$5 & -0.35 & 0.03 & -0.03 & 0.00 & -0.03 & -0.20 & 0.34 & -0.73 & -3.54 \\ 
& -0.33 & 0.19 & -1.45 & 1.29 & -0.43 & -0.34 & 1.10 & -1.19 & -3.01 \\
  1$|$6 & -11.48 & -0.05 & -0.02 & 0.00 & -0.16 & -0.31 & -0.14 & 11.33 & 10.69 \\ 
& -0.08 & -0.24 & -0.58 & 0.11 & -1.64 & -0.40 & -0.32 & 0.08 & 0.07 \\ 
  \hline
\end{tabular}
}
}

\hspace{1em}

\subfloat[Model with Council Dummies and Controlling for Salience]{
\resizebox{\textwidth}{!}{ 
\begin{tabular}{rrrrrrrrr}
  \hline
 & Constant & Outlier & Power & Net Beneficiary & Meeting Frequency & Presidency & Agriculture/Regional & B Points \\ 
  \hline
1$|$2 & 2.44 & -0.22 & 0.01 & 0.00 & 0.06 & 0.07 & -0.06 & -0.06 \\ 
 & 5.74 & -2.95 & 1.55 & 2.00 & 1.72 & 0.26 & -0.39 & -1.45 \\ 
  1$|$3 & -1.44 & 0.03 & 0.03 & -0.00 & -0.00 & 0.25 & 0.82 & 0.13 \\ 
& -1.94 & 0.21 & 1.75 & -0.17 & -0.00 & 0.63 & 3.35 & 2.20 \\ 
  1$|$4 & -1.02 & -0.05 & 0.03 & 0.00 & -0.06 & 0.18 & -0.12 & 0.08 \\ 
& -1.11 & -0.33 & 2.06 & 0.46 & -0.82 & 0.35 & -0.34 & 1.11 \\
  1$|$5 & -1.28 & 0.03 & -0.03 & 0.00 & -0.03 & -0.20 & 0.47 & 0.07 \\ 
& -1.43 & 0.20 & -1.45 & 1.28 & -0.42 & -0.36 & 1.51 & 0.85 \\ 
  1$|$6 & -0.12 & -0.05 & -0.01 & 0.00 & -0.16 & -0.31 & -0.11 & -0.00 \\ 
 & -0.10 & -0.27 & -0.55 & 0.11 & -1.68 & -0.41 & -0.26 & -0.01 \\
   \hline
\end{tabular}
}
}
\caption{Panel with Models 1 - 4. The coefficients are in first row for every contrast, and the t-values are reported in the seccond row}
\end{table*}


\renewcommand{\arraystretch}{1.5}
\begin{sidewaystable}[hp]
\centering
\resizebox{\textwidth}{!}{ 
\begin{tabular}{rrrrrrrrrrrrr}
  \hline
 & Constant & Outlier & Power & Net Beneficiary & Meeting Frequency & Presidency & Agriculture/Regional & Directive & Regulation & QMV & Unanimity & B Points \\ 
  \hline
1$|$2 & 3.56 & -0.22 & 0.01 & 0.00 & 0.04 & 0.06 & -0.02 & -0.40 & -0.57 & -0.41 & -0.59 & -0.06 \\ 
& 5.63 & -2.98 & 1.57 & 1.99 & 1.23 & 0.25 & -0.13 & -1.95 & -2.87 & -1.06 & -1.44 & -1.45 \\
  1$|$3 & -1.33 & 0.03 & 0.03 & -0.00 & -0.00 & 0.26 & 0.80 & -0.86 & -0.76 & 0.67 & -0.91 & 0.14 \\ 
 & -1.13 & 0.25 & 1.69 & -0.20 & -0.05 & 0.64 & 3.21 & -2.50 & -2.38 & 0.82 & -0.96 & 2.23 \\ 
  1$|$4 & -1.09 & -0.06 & 0.03 & 0.00 & -0.10 & 0.18 & -0.11 & 1.24 & 0.08 & 0.47 & -1.75 & 0.08 \\ 
 & -0.70 & -0.39 & 2.07 & 0.50 & -1.34 & 0.35 & -0.32 & 2.35 & 0.15 & 0.43 & -1.34 & 1.00 \\ 
  1$|$5 & -0.11 & 0.03 & -0.04 & 0.00 & -0.03 & -0.20 & 0.37 & -0.51 & -0.46 & -0.67 & -3.52 & 0.08 \\ 
 & -0.09 & 0.22 & -1.49 & 1.27 & -0.43 & -0.35 & 1.18 & -1.31 & -1.21 & -1.09 & -2.99 & 0.99 \\ 
  1$|$6 & -12.18 & -0.04 & -0.02 & 0.00 & -0.13 & -0.30 & -0.24 & 0.13 & 0.46 & 11.27 & 10.56 & 0.00 \\ 
 & -0.08 & -0.21 & -0.59 & 0.11 & -1.28 & -0.40 & -0.52 & 0.24 & 0.85 & 0.08 & 0.07 & 0.02 \\
   \hline
\end{tabular}
}
\caption{Full Model with Council Dummies. The coefficients are in the first row for every contrast, and the t-values are reported in the second row.}
\label{tab:full_final}
\end{sidewaystable}


\renewcommand{\arraystretch}{1.5}
\begin{sidewaystable}[!h]
\centering
\resizebox{\textwidth}{!}{ 
\begin{tabular}{rrrrrrrrrrrrrr}
  \hline
 & Constant & Outlier & Power & Net Beneficiary & Meeting Frequency & Presidency & Agriculture/Regional & Directive & Regulation & QMV & Unanimity & Salience & Outlier$\times$Power \\ 
  \hline
1$|$2 & 3.38 & -0.46 & -0.01 & 0.00 & 0.10 & -0.10 & 0.27 & -0.41 & -0.52 & -0.38 & -0.13 & -0.12 & 0.02 \\ 
& 4.37 & -2.77 & -0.26 & 3.12 & 2.54 & -0.33 & 0.72 & 6.15 & 0.09 & 2.61 & 3.70 & 4.84 & -0.00 \\ 
  1$|$3 & -2.04 & 0.50 & 0.07 & 0.00 & 0.03 & 0.19 & 0.91 & -0.62 & -0.38 & 0.62 & -1.48 & 0.10 & -0.07 \\ 
 & -1.72 & 1.75 & 2.48 & 1.57 & 0.26 & 0.40 & 3.21 & 0.27 & 0.00 & -0.05 & -0.43 & 1.17 & 0.69 \\ 
  1$|$4 & -1.55 & -0.55 & -0.01 & 0.00 & 0.01 & -0.14 & 0.50 & 1.11 & 0.09 & 0.62 & -1.09 & -0.02 & 0.04 \\ 
 & -1.59 & -1.67 & -0.31 & 1.36 & 0.81 & -0.22 & 1.14 & 1.12 & 0.00 & -0.08 & 0.95 & 0.57 & -0.08 \\ 
  1$|$5 & -0.22 & -0.26 & -0.09 & 0.00 & 0.04 & -0.21 & 0.44 & -0.57 & -0.41 & -0.67 & -3.04 & 0.02 & 0.03 \\ 
& -1.33 & -0.79 & -1.53 & 2.26 & 0.57 & -0.36 & 1.30 & 1.60 & 0.01 & -0.07 & 1.44 & 2.45 & -0.08 \\ 
  1$|$6 & -11.16 & -0.92 & -0.17 & 0.00 & -0.12 & -0.29 & 0.29 & -0.04 & 0.35 & 11.31 & 11.14 & -0.06 & 0.13 \\ 
& 0.44 & -2.18 & -2.16 & 0.64 & -1.22 & -0.38 & 0.66 & -0.42 & 0.01 & 0.06 & 0.10 & 0.40 & -0.07 \\ 
   \hline
\end{tabular}
}
\caption{Full Model with Council Dummies and Interaction Term. The coefficients are in the first row for every contrast, and the t-values are reported in the second row.}
\end{sidewaystable}

When inspecting the models it is clear that the effects of the main variables are robust to the inclusion of the procedural control variables. The effects of the variables also differ significantly across contrasts. The effect of being an outlier on the left-right scale in the Council, only show a significant effect when a member state government is choosing between voting yes or no. When the choice is between any other contrast the effect becomes insignificant. The effect of the power variable also shows a differentiated effect. Only when the choice is between voting yes or voting no and attaching a negative statement foes the variable become significant. One surprise is that the effect of being a net beneficiary from the EU is not more prominent, as it is usually claimed that this is one of the big splits within the Council. However in the full model there is only a significant effect when examining the contrast between voting yes or no. The meeting frequency and presidency variables fail to even come close any meaningful level of statistical significance. The dummy for agriculture and regional policies do achieve significance when comparing the yes/abstain contrast, but otherwise do not display any pattern. Finally the interaction term between the outlier and power variable fail to achieve any meaningful level of significance.

 Since the multinomial logit is non-linear, it is not possible to directly interpret the direction and magnitude of the effects from the coefficients. In table \ref{tab:effects} the effect of the outlier, power and net beneficiary variables for the Council  configurations with the most variance in the voting behavior. The effects are calculated based on the results from \ref{tab:full_final}. Since the interaction term did show any pattern it  has been left out of the interpretation. When calculating effects from non-linear models it is necessary to choose sensible levels at which to hold the control variables constant, while the variable of interest is allowed to vary. Here all continuous and count variables where held constant at their mean or median value, while categorical variables where held constant at the modal category. The effect calculated are thus for a government that are in the center of the left-right distribution, is from a medium sized member state and contributes slightly more to the EU than it receives back. The dossier is assumed to be a regulation, the voting rule is QMV and the dossier has been treated twice as a B point in the Council. Finally it is assumed that the council Configuration meets seven times a year. The effects represent the first difference in the variables when going from the minimum to maximum value on the variable of interest.

\begin{table}[htp]
  \centering
  \begin{tabular}{l l c c c}
  Variable & Choice &   Comp. & Agri/Pech & Empl. \& Social Affairs \\ \hline
\multirow{6}{*}{Outlier} & Yes & -.012 & -0.05 & -.02 \\
                                     & No  & .003 & .01 & .007 \\
                                     & Abstain & .001 & .003 & .01 \\
                                     & No + Statement & .002 & .005 & .002 \\
                                     & Abstain + Statement & .0008 & .006 & .004 \\
                                     & Yes + Statement & .006 & . 03 & .02 \\
 \multirow{6}{*}{Votes} & Yes & .008 & .03 & .016 \\
                                    & No & .0003 & .002 & .0004 \\
                                    & Abstain & .003 & .01 & .006 \\
                                    & No + Statement & -.007 & -.02 & -.012 \\
                                    & Abstain + Statement & -.0009 & -.008 & -.003 \\
                                    & Yes + Statement & -.004 & -.02 & -.007 \\
\multirow{6}{*}{Net Beneficiary} & Yes & .013 & .066 & .024 \\
                                                 & No & -.0008 & -.002 & -.001 \\
                                                 & Abstain & -.0009 &-.002 & -.002 \\
                                                 & No + Statement & .001 & .004 & .003 \\
                                                 & Abstain + Statement & -.001 & -.01 & -.004 \\
                                                 & Yes + Statement & -.01 & -.05 & -.02 \\ \hline
  \end{tabular}
  \caption{Effects}
  \label{tab:effects}
\end{table}

The effects presented in table \ref{tab:effects} might not seem very large. If we consider that a 2 point increase in the outlier variable is associated with an increase of 93 no votes, the effects become substantial. Considering the direction and size of the effects, the hypotheses relating to being a net beneficiary and being powerful are corroborated. Being a net beneficiary increases the number of yes votes in the Agricultural and Fisheries Council by 200 votes, while reducing the number of dissenting voters correspondingly. In the Competition and Employment \& Social Affairs Councils the effect is in the same direction, but markedly reduced. Indeed all effects show their strongest presence in the Agriculture and Fisheries Council, confirming that this Council configuration seems to be exhibit a hard bargaining approach to negotiations more often than other Council configurations. Surprisingly an the more an outlier a member state government is in the Council, that member state will exhibit an increase in the raw number of yes votes with a dissenting statement by a factor of 1.75 compared to a member state government on the median position. This is an effect which is consistently large across Council configurations and in terms of the increase in the raw number of this vote choice, it is by far the largest effect. Hence in conjunction with hard bargaining, we also see member state governments compromise on their positions, while at the same time signaling to their collegaues their disagreement. However this only happens when member states are outliers on the left-right scale. Furthermore the effect is present for both small and large member states. The effect that large member states vote yes more often than small  member states is most likely an enlargement effect, as the Eastern enlargement saw many small and medium sized member states enter the EU. It is well documented that they behaved more consensual in the first years of member ship. Furthermore the old large member states such as Germany, France, Italy, Spain and the UK have been in the sample longer, and compared to the new member states are thus represented by more votes. However this also implies that the estimate of the dampening effect that power has on dissent is conservative, and thus more likely to be larger. This provides corroborative evidence for hypothesis 2a, as one possible mechanism behind the effect of power is that large member states are able to move the final text closer to their ideal point. This effect also includes yes votes with an associated negative statement, hence alluring to a possible differentiated effect of Heisenbergs hypothesized ``favor bank'' \citep[p. 69]{Heisenberg2005}. 

The overall pattern that emerges from the models is one where receiving more from the EU than you give, and being powerful reduces dissenting behavior, whereas being a preference outlier decreases the number of yes votes, however the number of yes votes with a dissenting statement increases substantially. The first two effects can be explained by classic rational-institutionalist mechanisms, however the last effect is not so straightforward. From a rational-institutionalist perspective sending conflicting signals (voting yes, but saying no) is meaningful in signaling games where the players are trying to outsmart each other. In the present context the statement is attached after the vote, so it could not have been part of a bargaining strategy. From a logrolling perspective the behavior is also puzzling. There is no reason for an actor after a deal has been done, to then publicly state his/her opposition to this deal while still supporting it. From a logrolling perspective, once a deal has been done its done \citep{Warntjen2010}. From the perspective of diffuse reciprocity, such behavior makes more sense. Signaling do not necessarily have to be towards a well defined opponent, as the rational choice signaling games assume, but could also be aimed at future negotiations with the same partners. This notion of signaling to a future, unknown, negotiation, is only possible when a level of trust and reciprocity has been created in a group. This has been amply documented to be the case in the Council. Furthermore \citet{ElgstromJonsson2000} has shown that this type of behavior is consistent with a problem-solving style of bargaining, where there is a long shadow of the future. Hence the results reported here provide further support for the statement by Elgstrom and Jonnson, that the council is made up of:

\begin{quote}
  ``[...] a predominant problem-solving approavh with islands of conflictual bargaining.
\flushright{- \citealt[p. 697]{ElgstromJonsson2000}}
\end{quote}

\section{Conclusion}

In this paper a new data set was used which merged data from the PreLex data base and the Council monthly summaries. This allowed us to select cases that all had an element of conflict in them, based on the fact that non-controversial dossiers do not contain any meaningful information when studying decision making in the Council. The data was then disaggregated into each vote for each government in each member state of the EU. This allowed the use of a multinomial logit model to model the vote choice of the individual government. The implication being that studies who only focuses on member states risk biasing their analysis as governments change. That different governments from the same member state can differ a lot was witnessed in table \ref{tab:dissent}, where the Schroeder II government was among the ten most dissenting government in the EU in the 1999 - 2009 period, wheres the Merkel governments did not come close to the top. The explanation lies in the fact that the preference composition of the Council changes over time as old governments leave office and new governments enter office. Thus the Schroeder II government found itself an outlier in the Council (an average absolute deviation of 2.46 over its duration). The effect of being an outlier does not only lead to an increase in no votes and abstentions, but also to a substantial increase in dissenting statements attached to yes votes. An increase that is much more pronounced than the increase in no votes and abstentions. This type of behavior is difficult to reconcile with either a pure rational-institutionalist argument or a pure constructivist argument. In stead the answer is complex and lies between the two poles. Governments engage in both bargaining and problem-solving, however from af frequency point of view problem-solving is dominant. The fact that the main variable supposed to capture the effect of socialization did not show any effect, whereas the classic rational-institutionalist variables where significant, but had effects that where not predicted by the theory provides several points worthy of mentioning. First of all, the proxy for socialization was most likely a bad proxy, so it can be argued whether this paper successfully managed to find the pure rational-institutionalist effect of the outlier variable. This would explain the differentiated effects of the outlier variable. This, however, raises another point. Since a large part of the effect of the outlier variable was in an area of the dependent variable not measured in other studies, papers only using classic rational-institutionalist variables , without controlling for socialization effects, risks introducing bias into their models. 


\begin{table*}[ht]
\begin{center}
\begin{tabular}{rrrrrr}
  \hline
 & Mean & Standard Deviation & Max & Min & Observations\\ 
  \hline
Left-right & 5.36 & 1.43 & 7.75 & 2.59 & 13527\\ 
  Voting Power & 8.24 & 6.72 & 29.00 & 2.00 &13527 \\ 
  Net-beneficiary & -0.70 & 180.15 & 531.75 & -437.83 & 13527 \\ 
  Meeting Frequency & 7.84 & 3.49 & 22.00 & 3.00 & 13527\\ 
  B-point & 1.77 & 1.32 & 10.00 & 1.00 & 13527\\ 
   \hline
\end{tabular}
\end{center}
\caption{Summary Statistics for Count and Continuous Variables}
\label{tab:summary}
\end{table*}



