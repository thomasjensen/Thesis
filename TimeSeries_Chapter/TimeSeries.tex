
\chapter{Time and Diffuse Reciprocity}
\label{chapter:time}

\lettrine[lines = 3, findent = .5em]{V}{oting} behavior on the Council has for some time now been of great interest of scholars. Several studies has been carried out which look at different periods of voting in the Council. The common finding for all studies is that the level of consensus is remarkable high and stable over time \citep{MattilaLane2001,Mattila2004,Mattila2009, Heisenberg2005,
  Hayes-Renshawetal2006}. This research has greatly strengthened our knowledge on the Council. The claim that a consensus norm determines how voting behavior is structured in the Council is often made on the basis of the results from this literature. The consensus hypothesis, simply put, argues that the institutional setting of the Council furthers a high degree of trust among representatives of the member states. This in turn allows for member states to engage in diffuse reciprocity when sequencing exchanges \citep{ElgstroemJoensson2000, Lewis2000, Lewis2003, Heisenberg2005}. This is a very effective mechanism to promote long term cooperation, and is claimed to explain the high degree of consensual voting in the Council. So far this, however, has not been properly tested. For the consensus hypothesis time is a very important factor, as actors must have long time horizons for diffuse reciprocity to function. All studies so far has treated voting as being independent of time, and simply examined how the levels of consensual and conflictual voting change in different time periods, and how aggregate voting behavior can be used to examine the dimensionality of Council voting. However if the consensus hypothesis is correct then voting behavior in previous time periods will have a strong influence on voting behavior in the present time period. This implies that there is a long institutional memory in the Council which affects voting behavior. In technical terms voting in the Council can be viewed as a long-memory series \citep{Beck1992,Durr1992,CalderiaZorn1998,Pfaff2008}. This is a key implication which is exploited in this paper. Norms are notoriously difficult to study empirically \citep{EpsteinSegalSpaeth2001}. The effect of the long institutional memory is to keep conflictual voting behavior locked in at low levels, if these levels are determined by a consensus norm they should be linked together by an underlying unobserved cause. In this paper this underlying cause will be referred to as the consensus norm, and using recent developments developments in time series analysis it will be tested whether this cause is present or not. 

This chapter begins with a discussion of the consensus norm as an equilibrium induced by the social environment of the Council. This discussion is then tied to methodological issues of, showing how the consensus norm has important implications for the level of conflictual voting in the Council. It is demonstrated that a presence of a consensus norm should show itself as a cointegrating relationship between two types of conflictual voting. Since it has been argued that Eastern enlargement should have had a large effect on voting behavior in the Council; specifically we should have observed more conflictual voting, the impact of enlargement is examined on the overall levels of conflictual voting. An error-correction model of conflictual voting is estimated, and the implications are discussed. 


\section{Theory}
\subsection{Norms as Equilibria}
Often norm driven behavior is described i a loose sense of shared
expectations about appropriate behavior. This idea has in particular
been developed by
\citet{Bicchieri1990,Bicchieri1997,Bicchieri2006}. According to
Bicchieri a norm depends on an array of expectations and
beliefs. Indeed, following Bicchieri, we can define behavioural
regularity $R$ in population $P$ as a norm if:

\begin{enumerate}
  \item Almost every member of $P$ prefers to conform to $R$ on the
    condition (and only on the condition) that almost everyone else
    conforms too.
    \item Almost every member of $P$ believes that almost every other
      member of $P$ conforms to $R$.
\end{enumerate}

The conditions for when we can expect to
observe behavior based on beliefs and preferences about other actors
are a social setting with a limited number of actors, who repeatedly
interacts with each other \citep[32]{Bicchieri1997}. These are conditions which has been
stressed continuously in the literature on norms as being the
preconditions for normative behavior. It is often argued that the
insulation of Coreper combined with a high frequency of meetings in
different policy areas, with a set of participants that rarely change,
are the ideal preconditions for the rise of normative behavior
\citep{Lewis2003,Heisenberg2005}. It is argued that these
preconditions leads to an atmosphere of trust among the participants
and a long institutional memory. The time dimension of the
institutional environment in the Council has been referred as being
``essential'' for the development of a problem solving approach to
negotiations \citep{ElgstroemJoensson2000}. When the same group of
people work together for an extended period of time in a highly
institionalized setting the shadow of the future becomes very
important. In such a setting diffuse reciprocity is likely to
emerge. The reason being that the future is now important relative to
the present, and thus an obligation incurred in the present is likely
to be reciprocated in the future
\citep[p. 688]{ElgstroemJoensson2000}. The fact that an obligation do
not have to be reciprocated immediatly has immense effects of the way
interactions can be structured. It allows for sequential exchange
which reaches a long way into the future, which subsequently allows
the accumulation of ``debts'' and ``credit'' among actors. This
promotes long term cooperation very effeciently
\citep{Keohane1986}. Diffuse reciprocity is therefore the key
mechanism that allow for the existence of a stable norm of
consensus. Furthermore It is the sequencing of
exchange that makes diffuse reciprocity efficient in bringing about
consensus. That diffuse reciprocity is an important mechanism for
creating consensus in the Council has also been noted by Lewis who
writes:

\begin{quote}
``This diffuse form of reciprocity is reinforced weekly by the
horizontal nature of COREPER agendas, and the commonly cited
observation that when dealing with this many subjects  no one can
expect to be a \textit{demandeur} every time and still expect any kind
of understanding when their  needs are not being met.''
\flushright{ - \citealt[p. 268]{Lewis2000}}
\end{quote}


In this sense a consensus norm can also the thought of as a social
institution that provide information about how representatives of
member states are supposed to act when negotiating a dossier. There is
a natural affinity between this conception of the consensus norm and
the concept of equilibrium used in game theory. The consensus norm
represents an equilibrium reached among the member states in a highly
institutionalized and dense setting. No actor given current
information and its current position can do better by acting against
the norm on its own \citep[876]{CalderiaZorn1998}. If the consensus
norm indeed represents a social institution in the Council, then
member states will know when it is appropriate to voice their
disagreement, either through negative votes or abstentions. 

However, norms are not immutable, and they can and do change over
time, they can fail or in some circumstances break down completely
\citep{Bicchieri2006}. Changing the external conditions that lead to
the development of the current norm can have implications for how well
the social institution is applicable in the new environment. For the
Council of ministers in the EU eastern enlargement represents such a
change in the environment. The entry of a large group of new member states
into the Council represents an external chock to the consensus norm,
however if the norm is a true equilibrium we should expect to see a
stabilization of interactions among member states and the consensus
norm should reequilibrate itself. 

\subsection{Eastern Enlargement and the Norm of Consensus}
The Eastern enlargement has been viewed as a crucial event for the EU. With the addition of ten new member states it was widely predicted that the legislative machinery would have to change radically in order to be able to function efficiently. The reforms introduced in the Nice treaty was viewed as not sufficient to accomplish this, and it was widely predicted that a legislative gridlock would emerge \citep{TsebelisYataganas2002,KonigBrauninger2004}. In terms of a possible disruption of an equilibrium the Eastern enlargement is a prominent candidate. Ten new member states with heterogenous preferences, and from countries with very different traditions and economic conditions could spell the end of a normative equilibrium. This points to possible short and long term effects of Eastern enlargement. If behavior in the Council is based on a strong normative equilibrium induced by the social environment, then we should expect to see possible short term effects of enlargement, which would then disappear as the Council is reequilibrated. This is consistent with the adaptation hypothesis proposed by \citet{BailerHertzLeuffen2009}, in which the new member states are expected to behave out of equilibrium for a short period of time, however as they begin to adopt to the ways of doing business in the Council the hypothesis expects a return to the old equilibrium. In order to take this possible effect into account when testing for the consensus norm, the models are all estimated with enlargement dummies. 


\subsection{Voting Behavior and the Consensus Norm}
In this study the mechanism explicated above will be tested using Council voting records. In recent years we have seen a number of studies using Council voting records to make arguments about decision making in the Council. It is a fact observed in many studies that the
Council of the EU is remarkably consistent in adopting legislation
unanimously, even when the voting rule allows to use a qualified
majority (QMV).  In a series of papers Heisenberg, Hayes-Renshaw
et al., Mattila and Mattila \& Lane show that in any given year the Council adopts
between 78 and 90\% percent of legal acts unanimously 
\citep{MattilaLane2001,Mattila2004,Mattila2009, Heisenberg2005,
  Hayes-Renshawetal2006}. This is an effect which has been
remarkably consistent over several enlargements. Backed by a
series of case studies that all consistently show that there is a
large willingness to compromise in the Council, this has lead
some authors to argue that a norm of consensus is at work in the
Council \citep{Heisenberg2005, Warntjen2008}. This reading seem to
be confirmed by the fact that many researchers only find a very
weak dimensionality in the
Council. \citet{MattilaLane2001,Mattila2004,ZimmerSchneiderDobbins2005,HoylandHagemann2008}
all find that there are traces of left-right positioning and
redistributive politics in the Council. But none of the findings
are particularly strong. Thus it appears that in the absence of
strong ideological or re-distributional conflicts in the Council a
norm of consensus appear to dominate everyday decision making.  However a recent critique has resurfaced with regards to what can be learned from voting behavior. \citet{KonigJunge2009} argues that 
from the inspection of voting records alone it is difficult to
make any inferences about a norm of consensus. The voting records
might reflect that the Commission only introduces dossiers that it
know will find a majority among the member states, thus relieving
the member states of the necessity to use their no vote. Another
explanation for the high degree of unanimity could also be
log-rolling. Given the sectoral organization of the Council there
are many possibilities for log-rolling within the different policy
areas, and in COREPER  and Council sessions there is room for
log-rolling across policy areas. Thus voting records do not seem to a
useful dependent variable to analyze when explaining the causal effect of
the norm of consensus. 

One aspect of Council voting which escapes the critique raised by \citet{KonigJunge2009} is the time dimension. In normative voting behavior time plays a crucial role. As described above sequencing and diffuse reciprocity are key mechanisms for the functioning of a consensus norm. Logrolling is a very persuasive mechanism for reaching agreement across different issues and dossiers within the same time frame, however logrolling is less convincing as a mechanism for reaching agreement in different time periods. Simply put: if a logrolling agreement is reached across different time periods, there is a high incentive for defection for one or more of the parties. Furthermore logrolling must be specific, i.e. parties must know the exact positions and salience attached to the positions in order to make a trade. Thus under logrolling it is impossible for parties to make agreements about future events in which the content is not known. This implies that voting behavior under a logrolling mechanism is a short-memory process in which previous voting behavior should not substantially impact present voting behavior. As detailed above the consensus norm implies the opposite. Here diffuse reciprocity allows for a long time period to pass before a "debt" or "credit" is called. This  allows for a sequencing of exchanges over long periods of time involving dossiers which at one point might be unknown to the member states. If this view of negotiations in the Council is correct voting behavior should be long-memory process where voting behavior in the beginning of the process has effects throughout the whole series. This is a crucial distinction between the two mechanisms, and one that allows us to employ recently developed methods within the analysis of time series data to directly test for this. In section \ref{sec:coint} this will be dealt with in more detail.


\subsection{Abstentions and No-votes}
In the literature on the Council of ministers it is clear that the consensus norm exerts restraints on how much overt conflict is tolerated \citep{Lewis1998,Lewis2000,Lewis2003,Lewis2005,Heisenberg2005}, thus an appropriate indicator of how  a consensus norm works in the Council is the amount of conflict present. There are two basic forms that overt conflict in the Council can take, namely abstentions and no-votes. If a member state disagrees to a great extent with final version of a dossier it can vote no. The reasons for a no vote can be diverse and represent a host of different motivations. First, it is possible that important constituencies in a member state are adamantly opposed to the measures proposed in a dossier, and thus a no vote is motivated by signaling a governments commitment to represent their interests. Second, a no vote can represent an ideological motivated opposition by a member state government. Third, a no vote can be a part of a larger coalitional exercise in logrolling, where a no vote is provided in exchange for a no/yes vote on another dossier which is of more importance to a member state. Even though there are different motivations for voting no they all represent a breakdown of a norm of consensus. Abstentions are a different matter. They represent a milder form of conflict under some circumstances, while under other circumstances they are equivalent to no votes. Under a voting rule of unanimity, an abstention represents a mild form of dissent which allows a majority to proceed with a dossier. However, under QMV an abstention is formally equivalent to a no vote, as it is necessary to have at least 71\% percent yes votes, and an abstention makes it more difficult to reach this percentage \citep{HoylandHagemann2008}. This raises a puzzle. If an abstention is equivalent to a no vote under QMV why do we see member states engaging in both types of behavior when voting on QMV dossiers? One explanation is that even though the two types of voting behavior are formally equivalent, they represent different signals in Council negotiations. It is not controversial to claim that a no vote carries more weight when signaling to national constituencies. Therefore whenever signaling to national constituencies are the motivation for engaging on conflictual behavior, we should expect to see a member state voting no. Thus abstentions are difficult to reconcile with a signaling to home constituencies. However abstentions are a possible signal to other member states of the Council that a given country have issues with a dossier but it does not want to obstruct the negations. This is plausible in the light that often in the Council only one or two member states vote no or abstains from voting, thus almost never a blocking minority is reached. Under these circumstances the signaling value of an abstention can be directed at other member states in the Council, without jeopardizing the negotiations of a dossier.  Thus an abstention in the council can represent minor ideological differences, disagreements over details in the dossier and so forth. From this we can conclude that abstentions do represent a type of conflict in Council, however whereas no votes can be directed either at the other member states in the Council or key constituencies in the member state, abstentions can be seen as almost exclusively directed at other member states. In essence an abstention signals a small disagreement which is not larger enough to jeopardize the negotiations. However the disagreement is large enough for the member state to signal that the disagreement is not trivial. In sum both no votes and abstentions represent modes of conflict in the Council, it is, though, still an open question whether the occurrence of abstentions and/or no votes is the result of similar forces or constitute evidence of a behavioral norm.

\section{The Norm of Consensus and Cointegrating Relationships}
\label{sec:coint}
In this paper the consensus norm which has often been claimed to play a large role in decision making in the Council, is examined by treating abstentions and no votes as time series data on conflictual behavior in the Council. Basically there are two types of time series of interest to scholars, namely short- and long-memory series. A short-memory series is characterized the present value being virtually unaffected by a shock that occurred in the past. On the other hand a long-memory series is characterized by past chocks affecting the current value until the end of the series. Time series which are long-memory but which can be converted to short-term series by differencing once are integrated of order 1, and are referred to as having a unit root \citep{Pfaff2008}. 

In the Council there are several aspects which suggest that a possible consensus norm should be a long-memory process. The Council can be seen as a dense social environment where the history of interaction between actors plays a large role for how future negotiations are played out. As has been noted above the shadow of the future is a key mechanism for the development of a consensus norm, which in turn is  made possible by a dense social environment where actors continually interact with each other. This combined with a slow change in membership of the EU should lead the transfer of norms with regards to conflict to be stable across time. This is a point of view which is reflected in the common understanding that the norm of consensus developed as a solution to the empty chair crisis in the 1960s \citep{Heisenberg2005}. Therefore if we have a consensus norm present in the Council, this norm should structure to a high degree the amount of conflict tolerated. Thus a consensus norm is also manifested in both abstentions and no-votes,  and the degree to which they are accepted as appropriate behavior. If this is the case then we can follow \citet{CalderiaZorn1998} and decompose the series for absentions and no-votes into two parts:

\begin{align*}
N_t &= C_t + n_t \\
A_t &= NC_t + a_t
\end{align*}

where $A_t$ and $N_t$ are the number of no-votes and abstentions observed at time $t$, $C_t$ is the unobserved effect of the consensus norm at time $t$. $N$ is a multiplier for the effect of norms on abstentions such that the effect of norms on no-votes is normalized to 1. The terms $n_t$ and $a_t$ are stationary stochastic terms, with a zero mean, which represents other influences than the consensus norm on no-votes and abstentions. If this representation of how the consensus norm effects the level of conflict in the Council is accurate, and as suggested above that the two time series should have a long memory, then we can characterize the two series as cointegrated if we can find a linear combination of the two series which is stationary \citep{Durr1992,CalderiaZorn1998,Pfaff2008}. This allows us to determine whether the two series share a common component, in our case with regards to voting behavior in the Council this is a direct test of whether a norm of consensus is present or not. If we do not find a cointegration relationship between the two series then we must conclude that they are driven by different logics and there is not one underlying process which determine the levels of conflict in the Council. Thus if we find a cointegration relationship between abstentions and no-votes it is not a conclusive proof of the presence of a consensus norm, but it is very strong circumstantial evidence; and if we do not find a relationship it is strong proof against the presence of a consensus norm \citep[p. 881]{CalderiaZorn1998}. 

\section{Data, Methods and Results}
The data used in this paper comes from \citet{Hayes-Renshawetal2006} and \citet{Mattila2009}. Combining these two data sets provides us with data on voting behavior in the Council from 1998 to 2006. The data has been divided into quarterly counts of no votes and abstentions in order to capture whether a period of a high or low degree of no votes/abstentions has an effect on the rest of the series. Figure \ref{fig:time} show the two time series plotted form 1998 to 2006. 

\begin{figure}[htp]
\center
\includegraphics[scale=.6]{timeseries.pdf}
\caption{The Two Time Series: The plot show the series for abstentions and no votes plotted from 1998 to 2006.}
\label{fig:time}
\end{figure}

From a visual inspection it seems that the time series for abstentions exhibit a clear increase in mean and variance around 2004. The series for no votes could arguable seem to follow a slightly similar pattern with a sharp decrease after 2004, but then an increase in the mean after this drop. This provides preliminary evidence for the presence of non-stationary time series. Non-stationarity being a precondition for a long-memory series, this is cursory evidence for the presence of a consensus norm. To test for whether the two series are cointegrated we must first establish that each series is itself being integrated, if no integration  can be formally established for both time series there is no possibility for the series to be cointegrated. Thus a test for cointegration must begin testing both series for whether a unit root is present or not. If both series exhibit a unit root we estimate a cointegration regression for both series and test whether the residuals from the cointegration regressions are stationary. If the residuals are stationary it is strong evidence of a cointegration relationship between the two series. In order to determine the exogeneity of the series it is recommended to estimate an unrestricted vector auto-regression to determine which variable is dependent and which is independent. Finally we can estimate an error correction model to find the effects of exogenous variable(s) on the endogenous variable  \citep{Pfaff2008}.  

\subsection{Testing for Unit Roots and Conintegration in No Votes and Abstentions}
Table \ref{tab:ur_tests} reports the results from various tests of unit roots for no votes and abstentions. As the simple Dickey-Fuller test have low power in the presence of an autoregressive process the results for the augmented  Dickey-Fuller test (ADF) is reported. The lags has been selected according to Schwartz�s information criterion. The null hypothesis in the ADF tests is the presence of a unit root, and the alternative hypothesis is the absence of a unit root. Thus failure to reject the null hypothesis is here evidence for a unit root in the series. The \citet{KwiatkowskiPhillipsSchmidtShin1992} test is for whether the series is stationary or not. The null hypothesis is that the series is stationary, while the alternative is that the series could have a unit root. Hence in this case we want to reject the null hypothesis.

\begin{table}[htp]
\center
\begin{tabular}{l c c p{2.5cm}} \toprule
Tests & Abstentions & No Votes  & Critical Values (p $<$ .05) \\ \midrule
Augmented Dickey-Fuller Tests &  & & \\
 No Trend					& 0.7041 &  -0.9029 & -1.95 \\
 With Trend 				& -2.6681& -3.7089 & -3.50 \\
 (lags)					& (4)		& (1)		& -	\\
 Kwiatkowski et al Tests		&		&		& 	\\
 Lags = 0 					& 1.3022 & 0.459 	& 0.463 \\
 Lags = 2					&  0.8409 	&  0.3783 & 0.463 \\
Lags = 4					& 0.6072  & 0.3493  & 0.463 \\
Lags = 8					& 0.4108 & 0.279 	& 0.463 \\ \bottomrule

\end{tabular}
\caption{Tests for Unit Roots: The table show the tests for unit roots and stationarity of the series for abstentions and no votes.}
\label{tab:ur_tests}
\end{table}

Inspecting the table we find strong evidence that the series for abstentions has a unit root. We can reject the null hypothesis of stationarity for up to 8 lags, and we cannot reject the null hypothesis of a unit root in the ADF test. The picture is more murky when we look at the series for no votes. We cannot reject the null hypothesis of a unit root for a series with no trend, however when we include a trend in the test we can reject the null hypothesis. With regards to stationarity we can reject the null hypothesis at .1 level, and we are very close to rejecting it at the .05 level for zero lags, however as soon as we include a lag structure in the test we cannot reject the null hypothesis. This provides us with weak evidence for a unit root in the series, however we cannot reject  the presence of a unit root with great confidence. If we give the series for no votes the benefit of the doubt we can conclude that both series exhibit the presence of a unit root. This implies that both series exhibit a long memory and thus they behave in accordance with the prediction of the consensus norm hypothesis. 

Once we have established that the series are integrated by order of 1, the next step is to test for cointegration. \citet{EngleGranger1987,Durr1992,Pfaff2008} recommends a two-step procedure of first regressing each series on the other, and then as a second step test the residuals from each regression for stationarity. If the residuals are stationary we have evidence of a cointegration relationship between abstentions and no votes. Table \ref{tab:coint_reg} presents the results of the regressions. Two types of regression are estimated for each time series. In the first model we include no exogenous variables. In the second model a dummy variable for eastern enlargement has been included in order to examine whether the process is influenced by eastern enlargement. This allows us to judge whether the two series are cointegrated beyond the effect of Eastern enlargement. It also provides us with an initial look at how Eastern enlargement might have impacted voting behavior in the Council. 


\begin{table}[!ht]
\center
\begin{tabular}{ l D{.}{.}{2}D{.}{.}{2}D{.}{.}{2}D{.}{.}{2} } 
\toprule
 & \multicolumn{2}{c}{Without Enlargement} & \multicolumn{2}{c}{With Enlargement} \\ \cmidrule(lr{.75em}){2-3} \cmidrule(lr{.75em}){4-5}
  & \multicolumn{ 1 }{ c }{ Abstentions } & \multicolumn{ 1 }{ c }{ No Votes} & \multicolumn{ 1 }{ c }{ Abstentions } & \multicolumn{ 1 }{ c }{ No Votes } \\ \midrule
 %           & Model 1   & Model 2   & Model 3   & Model 4  \\ 
(Intercept) & 2.95      & 12.56   & 0.73      & 14.92  \\ 
            & (1.74)    & (3.22)    & (1.94)    & (3.12)   \\ 
No Votes         & 0.24   &    -       & 0.28    &     -     \\ 
            & (0.07)    &           & (0.07)    &          \\ 
Abstention         &        -   & 1.10   &      -     & 1.22  \\ 
            &           & (0.31)    &           & (0.30)   \\ 
Eastern Enlargement         &   -        &     -      & 4.17   & -10.00 \\ 
            &           &           & (1.91)    & (3.89)    \\
 $N$         & 36        & 36        & 36        & 36       \\ 
$R^2$       & 0.26      & 0.26      & 0.36      & 0.39     \\ 
Durbin-Watson  &1.51      & 1.28      &  1.78     &  1.60   \\  \bottomrule
 \multicolumn{5}{l}{\footnotesize{Standard errors in parentheses}}\\
\end{tabular} 
\caption{Cointegration Regressions: The models regress the two time series on each other and includes a dummy variable for Eastern enlargement.}
\label{tab:coint_reg}
 \end{table}
 
 Because of autoregression the significance values of the models cannot be trusted, however the point estimates are valid and since we have the entire population of votes in the period from 1998 to 2006 it is possible to make meaningful statements about this period based on the models. The effect of enlargement on voting behavior has been quite profound. The number of abstentions has gone up as a function of enlargement, while the number of no votes has decreased quite a bit. This indicates that the immediate effect of enlargement has been to reduce the overt levels of conflict but the small scale conflicts has increased. This indicates that member states wanted the Eastern enlargement to be a success and thus modified their behavior accordingly, however the levels of conflict within the Council did not decrease therefor we see an increase in the number of abstentions. 
 
 To test for whether the two series are cointegrated we use the residuals from the cointegration regressions and test whether they do not have a unit root and are stationary. Table \ref{tab:ur_tests_co} presents the results of ADF and KPSS tests for unit roots in the residuals from the cointegration regressions.
 
 \begin{table}[htp]
 \center
 \begin{tabular}{l p{2.2cm} p{2.2cm} p{2.5cm}} \toprule
 Tests & Residuals With Abstentions Dependent & Residuals With No Votes Dependent & Critical Values (p $<$ .05) \\ \midrule
 Augmented Dickey-Fuller Test & & & \\
 	Without Trend &-4.4865 & -3.5789 & -1.95 \\
 	With Trend & -4.3475 & -3.5039 & -3.50 \\
 Kwiatkowski et al Tests & & & \\
 Lags = 0 & 0.1024  & 0.2067  & 0.463 \\
 Lags = 2 & 0.1 & 0.1645 & 0.463 \\
 Lags = 4 &  0.1072  &0.15 & 0.463 \\
 Lags = 8 & 0.1687  &  0.1566 & 0.463 \\ \bottomrule
 \end{tabular}
 \caption{Tests for Unit Roots: The table show the tests for unit roots in the residuals from the cointegration regressions.}
 \label{tab:ur_tests_co}
\end{table}

From inspecting table \ref{tab:ur_tests_co} it is clear that we can reject the null hypothesis of a unit root and we cannot reject the null hypothesis of stationarity. Thus we can conclude that the residuals from the cointegration regressions are stationary with no unit roots. This is strong evidence of a cointegration relationship between abstentions and no votes, and it confirms the theoretical assumption that these two series are driven by the same underlying logic. This is a result that is further strengthened by the fact that we have taken Eastern enlargement into account in cointegration regressions. Controlling for this factor still lets us explain the cointegration of the two series through a common underlying factor. This is evidence of the presence of a consensus norm (assuming that the no vote series really does have a unit root). 

\subsection{An Error-Correction Model of the Consensus Norm}
Error correction models (ECMs) are a relatively new addition to the arsenal of methods available to the political scientist. In general ECMs are applicable when we have theory that dictates changes in a dependent variable which should be a function of both long and short term changes in the independent variables. The ECM modelling approach assumes that there exists an equilibrium state in which the levels of the two series are located vis-a-vis each other. This equilibrium can be disturbed by shocks, forcing the series wider apart (or closer together) than normal for the equilibrium state. This "error" in the equilibrium is corrected over time as the process finds a new level consistent with the equilibrium state \citep[p. 186]{Durr1992}. If the theory is correct we would expect to find an equilibrium in which the consensus norm induces an equilibrium and thus locks the levels of abstentions and no votes into a low level which is stable and robust to exogenous shocks.  In their seminal paper \citet{EngleGranger1987} introduce a two step process through which we can estimate an ECM. The first step is to estimate the cointegrating vector. This can be done by regressing the series on each other and examining the residuals for unit roots. If the residuals are stationary then we have the cointegrating vector. This is what we did above to establish whether two series are cointergrated. The second step in the procedure is to formulate an ECM model. The general specification of an ECM model follows the Engle-Granger representation theorem and has the following general form:

\begin{equation}
\Delta y = \beta_0 + \gamma_1 \hat{z}_{t-1} + \sum^K_{i = 1}\beta_{1,i} \Delta x_{t-i} + \sum^L_{i=1} \beta_{2,i} \Delta y_{t-i} + \epsilon_{1,t}
\end{equation}

\begin{equation}
\Delta x = \beta_0 + \gamma_1 \hat{z}_{t-1} + \sum^K_{i = 1}\beta_{1,i} \Delta y_{t-i} + \sum^L_{i=1} \beta_{2,i} \Delta x_{t-i} + \epsilon_{2,t}
\end{equation}

In the above equations $\hat{z}_t$ is the error from the cointegration regressions and $\epsilon_{1,t}$ and $\epsilon_{2,t}$ represent white noise processes. The equations basically state that changes in the dependent variable are explained by their own history lagged changes of the independent variables, and the error from the cointegration regressions. The estimated cointegrating vector from the first step represents the amount of error in the long-memory equilibrium in the previous time period. The value of the coefficient represents the speed of reequilibration when the system has been victim to a shock, and should always be negative. If the sign of the coefficient estimated from the residuals of the cointegration regressions is positive it means that there is no equilibrium in the system, which implies that shocks to the system are never adjusted to a new equilibrium. For these reasons the residuals from the cointegration regressions are also referred to as the error correction term. The advantage of the ECM models is that independent variables incorporated into the cointegration regressions have their effects felt over a longer period of time. Thus we can estimate the long-run impact of a change in the consensus norm, in our case what would happen when a series of new member states are introduced, by examining the $\gamma$ coefficient.  We can write our ECM model as follows:

\begin{equation}
\Delta Abstentions = \beta_0 + \gamma \hat{z}_{t-1} + \textrm{ \textit{lags of} } \Delta No Votes, \Delta Abstentions
\end{equation}

and

\begin{equation}
\Delta No Votes = \beta_0 + \gamma \hat{z}_{t-1} + \textrm{ \textit{lags of} } \Delta No Votes, \Delta Abstentions
\end{equation}

Furthermore we can unpack the error correction term by using the results from table \ref{tab:coint_reg} and rewrite it as a function of no votes, abstentions and Eastern enlargement. With abstentions as the dependent variable we can rewrite the error correction term as:

\begin{equation}
\gamma \hat{z}_{t-1} = Abstentions_{t-1} - 0.28*No Votes_{t-1} - 4.17*Enlargement_{t-1}
\end{equation}

With no votes as the dependent variable we can rewrite the error correction term as:

\begin{equation}
\gamma \hat{z}_{t-1} = No Votes_{t-1} - 1.22*Abstentions_{t-1} + 10.00*Enlargement_{t-1}
\end{equation}

Thus it is possible to express the effect of the error correction term as a function of enlargement, holding the level of no votes and abstentions constant. The first step of testing for whether we have an underlying normative equilibrium that determines the levels of abstentions and no votes is to estimate an ECM for the two series. If the series are cointegrated at least one of the ECMs should have a significant and negative coefficient. This is indicative of granger causation \citep{Granger1988}. Hence even if there is no short term effect of enlargement, there might still be a long term effect of enlargement that works through the error correction term. Table \ref{tab:ecm} show the results from the ECMs

\begin{table}[!ht]
\begin{tabular}{ l D{.}{.}{2}D{.}{.}{2}D{.}{.}{2}D{.}{.}{2} } 
\hline 
  & \multicolumn{ 1 }{ c }{ $\Delta$NoVotes } & \multicolumn{ 1 }{ c }{ $\Delta$Abstentions} & \multicolumn{ 1 }{ c }{ $\Delta$NoVotes  } & \multicolumn{ 1 }{ c }{ $\Delta$Abstentions  } \\ \hline
 %               & Model 1  & Model 2  & Model 3  & Model 4 \\ 
(Intercept)     & -0.76    & -0.22    & -0.30    & -0.43   \\ 
                & (1.81)   & (1.06)   & (2.23)   & (1.31)  \\ 
$\Delta$Abstentions$_{t-1}$& -0.28    & -0.65 ^* & -0.28    & -0.65 ^*\\ 
                & (0.28)   & (0.16)   & (0.28)   & (0.17)  \\ 
$\Delta$NoVotes$_{t-1}$ & -0.46 ^* & 0.06     & -0.47 ^* & 0.06    \\ 
                & (0.14)   & (0.08)   & (0.14)   & (0.08)  \\ 
$\gamma \hat{z}_{t-1} $    & 0.84 ^*  &          & 0.84 ^*  &         \\ 
                & (0.19)   &          & (0.19)   &         \\ 
$\gamma \hat{z}_{t-1}$     &          & 0.91 ^*  &          & 0.91 ^* \\ 
                &          & (0.23)   &          & (0.23)  \\ 
Eastern Enlargement$_{t-1}$         &          &          & -1.45    & 0.66    \\ 
                &          &          & (3.93)   & (2.31)   \\
 $N$             & 34       & 34       & 34       & 34      \\ 
$R^2$           & 0.62     & 0.51     & 0.62     & 0.52    \\ 
adj. $R^2$      & 0.58     & 0.47     & 0.57     & 0.45    \\ 
Resid. sd       & 10.53    & 6.18     & 10.69    & 6.28     \\ \hline
 \multicolumn{5}{l}{\footnotesize{Standard errors in parentheses}}\\
\multicolumn{5}{l}{\footnotesize{$^*$ indicates significance at $p< 0.05 $}} 
\end{tabular} 
\caption{ECM: The table show the results from the four ECMs.}
\label{tab:ecm}
 \end{table}

As the results from table \ref{tab:ecm} clearly show the error correction terms does enter significantly, however they are positive. Thus there is no underlying equilibrium that dictates how conflictual behavior in the Council. In stead the results indicate that conflictual behavior fluctuates according to shocks in the level of abstentions and no votes. These shocks are amplified until a new shock is felt, which might go in the opposite direction. The key result here is that conflictual behavior in the Council does not tend towards an equilibrium. This confirms that the series of no votes do not have a unit root, and thus cannot be seen a long-memory series.  

\section{Discussion and Conclusion}
The paper started by theorizing the consensus norm as an equilibrium induced by the dense social environment surrounding negotiations in the Council. The large shadow of the future in such a setting makes it possible to engage in diffuse reciprocity, which is a key mechanism for the consensus norm to function. One direct implication of this is that voting behavior in the Council should be seen as long-memory processes. As member states engage in negotiations they build up debts and credits and we should expect conflictual behavior to stabilize at a low level. This is a very attractive hypothesis. In order to test this hypothesis this paper collected data on voting behavior in the Council from 1998 to 2006, divided the data into quarters and counted the number of abstentions and no votes in each quarter. The two time series was then tested for unit roots and used to estimate four ECMs to test for whether there was a normative equilibrium that governed the level of conflict in the Council. The results are not favorable to the hypothesis. The abstention time series has a unit root, and can be characterized as a long-memory series, however the series for no votes do not behave the same way. Using standard tests for unit root it was not possible to either confirm or reject the presence of a unit root for the no vote series. However the results from the ECMs established that there is no underlying equilibrium that governs the levels of no votes and abstentions, thus confirming the suspicion that the no votes series is not a long-memory process. This has several implications for the literature on the consensus norm. What the results indicates is that there is no norm that governs how the level of conflict fluctuates in the Council. These pieces of evidence corroborates the logrolling argument where the voting behavior is determined by whatever deal a member state is engaged in at the moment. This does not prove that no consensus norm exists, however if it exists it does not structure the level of conflict in the Council. Lewis has claimed that the consensus norm will break down whenever a member states key interests is at stake \citep{Lewis2000}. However if this is the case then it is very difficult to separate normative voting behavior from overlap of preferences and low saliences. From a normative point of view this is not an attractive explanation of why we see the conflict patterns that we do. From a rational point of view there are two explanations which are consistent with the analysis presented here. First, It is a fact that we have a high degree of unanimous voting in the Council. Hence it is likely that the seemingly random pattern of conflict which we observe is due to breakdowns in logrolling negotiations. This would happen whenever member states would misjudge the position and/or level of salience that their partners attach to issues. In a highly socialized environment like the Council we would not expect this to happen systematically, and thus the breakdowns we do observe are random occurrences that represent a momentary lapse in member states bargaining behavior. Second, it is possible that the pattern we observe is due to the type of dossiers submitted by the Commission. It has often been argued that the Commission submits dossiers in a strategic manner in order to optimize the adoption rate \citep{Steunenberg1996, Hix1999, Tallberg2002}. If this is the case then the pattern could also be explained by small unsystematic errors of judgement by the Commission. On average the Commission could be very successful in submitting dossiers which are not controversial for most member states. However once in a while a random error might be made from the Commissions side which leads it to submit a dossier which is problematic for some member states. 

In sum, the analysis presented here does pose some hard questions for the claims that negotiations in the Council is guided by normative actors. No definitive proof against the consensus norm has been presented, but there is strong circumstantial evidence that no norm guides the level of conflict in the Council. If a norm of consensus only guides the affirmative votes then it must be established that at least in some cases the unanimity does not reflect logrolling, an overlap of preferences, or low saliences. This is a very difficult task, but it is necessary if the consensus norm hypothesis is to continue to be relevant for studies of the Council.  

