\documentclass{article}

\usepackage{natbib}
\usepackage{booktabs}
\usepackage{graphicx}
\usepackage{array}

\title{Reply To Comments: Theory Chapter}
\date{30/08/2011}

\begin{document}

\maketitle

\section{Logrolling and the Spatial Model}

The following comments were made with regards to the spatial models present:

\begin{quote}
  ``Why don't you use established models for this? Where is the idealpoint of the actors?''
\end{quote}

\begin{quote}
  ``I still have some doubts about your interpretation of logrolling in the spatial model.''
\end{quote}

The models presented are standard spatial models in one and two dimensions (see \citealt{HinichMunger1997}). In the two dimensional model the ideal points of the actors are the SQ and the location of $p_1$ and $p_2$ on the respective dimensions. This can be seen by following the curves which all pass through the SQ and tops at the locations of $p_1$ and $p_2$

In the spatial model time is not a factor. This is equivalent to  making the assumption that time does not matter. What matters in the spatial model is the preferences and saliences of actors. It is assumed that all actors know each others preferences and saliences perfectly (for a detailed exposition of how actors and saliences are related in multiple dimensions see \citealt[chapters 3 and 4]{HinichMunger1997}). If we make the uncontroversial assumption that actors cannot know their preferences perfectly with regards to future issues or dossiers, then according to the spatial model no deal can be made. If we assumem that actors can know their preferences and saliences perfectly over time and that voting takes place at different time points, then there is always an incentive for the actor whose part of the deal was fulfilled first to default on the bargain during the second vote. Only if we assume that agreements are binding can this be avoided, however then we move into the area of cooperative game theory, whereas spatial models are decidedly uncooperative. 

\section{Scope Conditions}

The following comment was made with regards to the inclusion of insulation from domestic scrutiny as a scope condition on page 12:

\begin{quote}
  ``not clear why this is a scope condition for diffuse reciprocity. domestic public scrutiny may or may not help depending on the positions and attitudes of domestic actors. if the government has an incentive to defect (against the preferences of the  domestic public), public scrutiny may even help.''
\end{quote}

The inclusion of this scope condition is mainly inspired from \citet{Lewis2005} who includes two scope conditions in his discussion of socialization, namely issue density/intensity and insulation. In the context of diffuse reciprocity it is important that actors can compromise, even on issues that might be sensible domestically. Public scrutiny can make it difficult to engage in compromise as governments risk alienating important domestic constituencies. When engaging in social exchange all parties potentially have an incentive to defect, however the shadow of the future is what keeps them from actually defecting. Indeed public scrutiny might help, however it is not necessary, as long as the shadow of the future is strong. Moreover if public scrutiny can lead to governments becoming unwilling to engage in compromise, then the presence of public scrutiny can prevent the advent of a system of diffuse reciprocity. The point is well taken that this was not clear from the text and the paragraph has been modified accordingly.

The following comment was made with regards to the number of scope conditions listed on page 12:

\begin{quote}
 `` Can you name further scope conditions of diffuse reciprocity? I don't think that these are the only ones. For instance, under the conditions of extremely asymmetric interdependence, some actors might not need to care about their reputation.''
\end{quote}

This point is well taken, and this condition has been added to the text.

\section{Signaling and Voting Records}

Some concern has been expressed whether 

\begin{quote}
  ``[...] government representatives really check the voting records to update themselves on diffuse reciprocity?''
\end{quote}

And more pointedly:

\begin{quote}
  ``I don't buy this because if there is really a deal people will remember and I do not see negotiators running behind their colleagues with Council summaries of voting behavior.''
\end{quote}

The point to be made with using voting records in conjunction with signaling, is that the voting record acts an insurance against potentially forgetful colleagues. Hence the implication is not that representatives will continuously check the voting records to see who did what, but rather that representatives have the possibility to point to the record, if the need should arise. This has been clarified in the text. 

With regards to the interpretation of voting behavior the following comment was made:

\begin{quote}
  ``Make clear what the status of these propositions is: assumptions, hypotheses? any anecdotal or other evidence to support initial plausibility?''
\end{quote}

Unfortunately I have not been able to find any case studies that deal with voting in the Council, that could lend supporting evidence. I have modified the text to make it clear what is based on assumptions and what is based on fact. 

Moreover:

\begin{quote}
  ``why not say that a no is also a strong signal to other Council members that the government dislikes the proposal more than under abstention?''
\end{quote}

Under QMV no votes and abstentions have the same effect, both will make it harder to reach a qualified majority. This, I assume, most government representatives in the Council are aware of. Hence from the perspective of a government representative in the Council there is not much of a difference between the two types of votes. The puzzle is, then, if votes only signal dissatisfaction with a given dossier, why do governments sometimes use abstentions, and sometimes vote no? Both votes have the same effect. The implication is that we cannot explain the variation in the use of these votes by referring to the different degrees of dissatisfaction they might represent, as in the Council both votes, in praxis, have the same effect. Under unanimity there is a difference, however in cases where a member state will vote no the dossier is returned to the Commission, hence we do not observe these cases. The big assumption I make is that this lack of distinction between no votes and abstentions is not readily seen among domestic constituencies. Hence towards a domestic constituency a no vote is a stronger than an abstention. 

\section{Table 1.1: Types of Signaling in the Council}

With regards to the table, the following comment was made:

\begin{quote}
  ``what the table does not contain is an interpretation of the signal with regard to diffuse reciprocity. which behavior makes a higher deposit in the favors bank than others? and: here you don't distinguish between the voting rules.''
\end{quote}

In the discussion on pages 12 - 14, the argument was made that the votes under both QMV and unanimity can be interpreted in the same way. Hence in terms of signaling behavior there is no difference between the two voting rules. In the paragraph below the table, I argue that only the third choice category counts towards diffuse reciprocity. The other two categories represent deviations from diffuse reciprocity. I have added some more text trying to make the point more clear. 

It was also suggested to expand the table with interpretations facetted by voting rule, spatial model, domestic audience and diffuse reciprocity. Such a table might look like Table \ref{tab:expanded}. In this table the row for no votes are left empty under unanimity, as we do not observe these cases in the Council voting records. Cells are left empty when there is no implication based on the discussion in the chapter. 

\begin{table}[htp]
\setlength{\extrarowheight}{10pt}
  \centering
  \resizebox{\textwidth}{!}{ 
    \begin{tabular}{l p{2cm} p{2cm} p{2cm} p{2cm} p{2cm} p{2cm}} \toprule
      & \multicolumn{3}{c}{QMV} & \multicolumn{3}{c}{Unanimity} \\ \cmidrule(lr{.75em}){2-4} \cmidrule(lr{.75em}){5-7}
      & Spatial Model (2D) &Domestic Audience & Diffuse Reciprocity & Spatial Model (2D) & Domestic Audience & Diffuse Reciprocity \\ \cmidrule(lr{.75em}){2-2} \cmidrule(lr{.75em}){3-3} \cmidrule(lr{.75em}){4-4} \cmidrule(lr{.75em}){5-5} \cmidrule(lr{.75em}){6-6} \cmidrule(lr{.75em}){7-7}
      No Vote & No deal could be made / dissatisfaction with the outcome & Defense of national interests & Defense of national interests & - & - & - \\
      Abstention & - & - & Signal to Council colleagues, dissatisfaction with outcome & - & - & Signal to Council colleagues, dissatisfaction with outcome \\
      Yes + Statement & - & - & Engaging in diffuse reciprocity & - & - & Engaging in diffuse reciprocity \\ \bottomrule
    \end{tabular}
  }
  \caption{To be determined}
  \label{tab:expanded}
\end{table}

The table has a lot of empty cells, which stems from the fact that the spatial model do not make any predictions with regards to abstentions and yes votes with a dissenting statement. From a domestic audience perspective only the no vote category is relevant, whereas from a diffuse reciprocity point of view most cells are full. The interpretations are the same across voting rules, hence the table could be collapsed into table 1.1 in the chapter. 

\section{Signaling and Game Theory}

It has been suggested that the game theoretic approach to signaling should be explored in more detail in this chapter. I have spend a great deal of time thinking of how one could devise a signaling game that captures the logic of voting as signaling devices. Signaling games consist of the following components:

\begin{enumerate}
\item Several actor types and a probability distribution over these types
\item A set of feasible messages
\item A set of feasible actions
\item A set of payoffs
\end{enumerate}

The structure of a signaling games is as follows:

\begin{enumerate}
\item Nature draws a type $t$ for the sender of the signal, according to the probability distribution.
\item The sender observes his/her type, and chooses a message from the set of feasible messages.
\item The reciever observes the message, but not the type of the sender. 
\item The receiver chooses an action from the set of feasible actions.
\item Payoffs are given by according to the type of the sender, the message of the sende and the action of the reciever.
\end{enumerate}

The problem with fitting diffuse reciprocity and voting as a signal into this framework, is that the signaling in diffuse reciprocity is aimed at future interactions, and not targeted at a specific actor (unless a majority is tought of as a unitary actor). In signaling games the signals are send before any action is taken, in the theory of diffuse reciprocity the signals are send after it is clear what the outcome of a negotiation will be. Finally it is not clear what the different actor types should be. For these reasons I gave up trying to create a signaling game for voting behavior. 

\section{Diffuse Reciprocity and Social Norms}

The following comments were made with regards to the discussion on social norms and diffuse reciprocity:

\begin{quote}
  ``Here you seem to equate diffuse reciprocity with social norms. But diffuse reciprocity is only one type of social norm.''
\end{quote}

\begin{quote}
  ``I am not sure whether diffuse reciprocity is really a norm. Doesn't it present a mechanism behind explaining specific behavior that at first seems strange from a rationalist point of view?''
\end{quote}

These comments are well taken! I have added some text clarifying that diffuse reciprocity is only a case of a social norm. Diffuse reciprocity is the result of a series of social exchanges, which, when accumulating, results in a web of relations based on credits and debits. Diffuse reciprocity is essentially the praxis of engaging in compromise, in the expectation of possible future gains. Actors only prefer to engage in this praxis if almost everyone else does as well, otherwise the risk of defection is to great. If they believe that almost every other actor actually follows the praxis of diffuse reciprocity, then they will as well. Hence the mechanism that leads to the rise of diffuse reciprocity is social exchange, and diffuse reciprocity as a norm embodies a certain praxis. 

\section{Other Comments}

The following comment was made:

\begin{quote}
  ``Wouldn't you need to say more about signaling and how it affects reciprocity, learning, etc. for a complete theory chapter?''
\end{quote}

I have left the hypotheses to be developed in each of the empirical chapters, based on the framework from the theory chapter. Hopefully in the empirical chapters this is done satisfactorily. 



 

\bibliography{/Users/thomasjensen/Documents/library}
\bibliographystyle{apsr}

\end{document}
